\section{Second LAB}
\cite{LAB2}


\subsection{OpenSSL Commands}

\begin{table}[H]
	\centering
    \begin{tabular}{|p{6cm}|p{3cm}|p{7cm}|}\hline
        \rowcolor{gray!30}
		\textbf{Command} & \textbf{Description} & \textbf{Options} \\ \hline
		\textcolor{Blue}{\texttt{man openssl <command>}} 
			& 
			& 
			\\ \hline
        \texttt{openssl enc} 
			& Allows the encryption and decryption of data with several symmetric cipher
        routines. 
			& -help; -ciphers; -p
				\newline -<algorithm>; -nopadK;
				\newline -K <hexKey>; -iv <hexVector>
				\newline -in <inputFile>; -out <outputFile>
				\newline -iter <n>; -pbkdf2; -nosalt
				\newline -e (default); -d; 
				\\ \hline
        \texttt{openssl rand <numBytes>} 
			& Generates nBytes pseudo-random data.
			& -hex
				\newline -out <outputFile>
			\\ \hline
        \texttt{openssl genrsa <numBits>} 
			& Performs simple asymmetric (key pair) operations with the RSA algorithm.
			& -out <outputFile>
			\\ \hline
		\texttt{openssl rsa} 
			& To manage and use the RSA keys in cryptographic operations.
			& -in <inputFile>; -out <outputFile>
				\newline -text; -noout
				\newline -pubin; -pubout
			\\ \hline
		\texttt{openssl ecparam} 
			& To manage and manipulate the EC algorithm parameters.
			& -list\_curves
				\newline -name <curveName>; -genkey
				\newline -out <outputFile>
			\\ \hline
		\texttt{openssl ec} 
			& To manage and manipulate the EC algorithm keys.
			& -in <inputFile>; -out <outputFile>
				\newline -pubin; -pubout
				\newline -text
			\\ \hline
		\texttt{openssl pkeyutl} 
			\newline\newline \textbf{Supported algorithms:} RSA, DSA, Diffie-Hellmann and Elliptic Curve. \textbf{The order in which the parameters are passed is important.}
			& Performs asymmetric encryption/decryption, signature/verification, and key exchange, by using various asymmetric algorithms.
			& -encrypt; -decrypt; -sign; 
				\newline -verify; -verifyrecover
				\newline -in <inputFile>; -out <outputFile>
				\newline -pubin; -inkey <keyFile>
				\newline -sigfile <signatureFile> (verify)
			\\ \hline
		\texttt{openssl dgst <inputFile>} 
			& Allows to calculate the digest of data using different algorithms.  
			& -list
				\newline -<algorithm>; -out <outputFile>
			\\ \hline
		\texttt{openssl speed} 
			& Measures the performance of the various algorithms implemented by OpenSSL
			& -evp (ctr)
			\\ \hline
    \end{tabular}

    \caption{\texttt{openssl} commands}

    \label{tab:openssl}
\end{table}

\clearpage
\subsection*{Insights}
\begin{itemize}
\item 1 Byte = 2 HEX characters. 
\item In order to decrypt a file you need to know: iv, K and cipher algorithm.
\item In practice, if you have an N-Bytes RSA key, you can perform successfully encryption/decryption operations with OpenSSL only if the (plaintext) data is at most N-11 bytes long.
\item RSA-encrypt \textrightarrow public key.
\item RSA-decrypt \textrightarrow private key.
\item RSA-sign \textrightarrow private key.
\item RSA-verify \textrightarrow public key.
\item The \texttt{pubin} parameter is used to specify that the input key it has to be a public key.
\end{itemize}

\subsection{Utility Commands}
\begin{table}[H]
	\centering
    \begin{tabular}{|p{5cm}|p{6cm}|p{5cm}|}\hline
        \rowcolor{gray!30}
        \textbf{Command} & \textbf{Description} & \textbf{Options} \\ \hline

		\texttt{systemctl [options] ssh} 
		& Must be enabled on the Receiver
			\newline \textbf{Remember to stop it at the end}.
		& start ; stop ; restart
			\newline enable ; status 
		\\ \hline

		\texttt{scp <user>@<ipReceiver>\newline:<dirFullName>} 
		& Transfers a file to the specified user’s directory
		& start ; stop ; restart
			\newline enable ; status 
		\\ \hline

		\texttt{openssl rand -out <outputFile> <numBytes>} 
		& Creates a file numBytes long.
		&
		\\ \hline

		\texttt{time <openssl\_command>}
			& Measures the elapsed time of a command.
			&
			\\ \hline

		\texttt{expr <arg1> <basicOperation> <arg2>}
			& Performs basic operations.
				\newline Such as: \textbackslash* / + - 
			& 
			\\ \hline
		
		\texttt{wget <URL>}
			& For non-interactive download of files from the Web.
			& 
			\\ \hline

		\texttt{atril <fileName> \&}
			& A simple multi-page document viewer.
			& 
			\\ \hline

		\texttt{sha1sum}
			& Easy computation of the hash of one or more files.
			& 
			\\ \hline

		\texttt{hashdeep <file/dirName>}
			& Easy computation of the hash of one or more files. Processes recursively the files contained in a directory with a chosen algorithm.
			& -r; -c <dgstAlgorithm>
				\newline -m (match)
				\newline -x (negative match)
				\newline -k <fileName> (for m or x)
			\\ \hline

	\end{tabular}
	\label{tab:utilityCommands}
	\caption{Utility Commands}
\end{table}
\subsection*{Insights}
\begin{itemize}
	\item File-transfer protocol: enable ssh server on the receiver (remember to stop it at the end), send the file from the mittent with scp tool.
	\item Command: \texttt{scp <fileName> <user>@<ipReciever>:/home/<user>/Desktop}
	\item \textcolor{Blue}{\texttt{scp}: in my case user=Alice or Bob, with their ip provided from \texttt{ifconfig}, password=0000}
\end{itemize}

\subsection*{Operations with Digests}

\begin{lstlisting}[style=bashStyle]
	#generate hashes for the files within the "tree" directory and save them to hash_list
	hashdeep -c sha256 -r tree > hash_list
	#check for differences on the same files
	hashdeep -c sha256 -r -x -k hash_list tree
	
\end{lstlisting}

%----------------------------------------

\clearpage

\subsection*{Operations on Key Pair}

\begin{lstlisting}[style=bashStyle]
	#create a key pair and save them to a file
	openssl genrsa -out rsa.key.Alice 2048 
	#read the key file
	openssl rsa -in rsa.key.Alice -text 
	#extract only the public key and save it to a file
	openssl rsa -in rsa.key.Alice -out rsa.pubkey.Alice -pubout

	#encrypt a plain text with a public key
	openssl pkeyutl -encrypt -in plain -out encRSA -pubin -inkey rsa.key.Alice 
	#decrypt a cipher text encrypted with RSA, knowing the private key
	penssl pkeyutl -decrypt -in plain.enc.RSA.for.Alice -inkey rsa.key.Alice

	#sign a file ("plain") using the private key of Alice
	openssl pkeyutl -sign -in plain -inkey rsa.key.Alice -out sig.Alice
	#verify the signature (on the file "plain") using the public key of Alice
	openssl pkeyutl -verify -in plain -pubin -inkey rsa.key.Alice -sigfile sig.Alice

	#generate a SECG curve over a 192 bit prime field and save it to a file
	openssl ecparam -name secp192k1 -genkey -out ec.key.Alice
	#extract the ec public key from a file and save to another file
	openssl ec -in ec.key.Alice -pubout -out ec.pubkey.Alice

	#sign a file ("plain") with  ECDSA and save the signature to a file
	openssl pkeyutl -sign -in plain -inkey ec.key.Alice -out ecsig
	#verify the signature of the file signed with ECDSA
	openssl pkeyutl -verify -in plain -pubin -inkey ec.pubkey.Alice -sigfile ecsig
\end{lstlisting}

\subsection*{Symmetric Algorithms Performances}

Be aware: The real time reported by the time command in the table \ref{tab:symmPerformance} refers to the elapsed wall clock time — the total time from when the command starts executing to when it finishes.
\newline Creating files: \texttt{openssl rand -out <outputFile> <numBytes>}.
\newline Measuring elapsed time: \texttt{time <opensslEncryptionCommand>}.
\begin{table}[H]
		\centering
		\begin{tabular}{|p{3cm}|p{2cm}|p{2cm}|p{2cm}|p{2cm}|}\hline
	\rowcolor{blue!10}
	& 100 B & 10 kB & 1 MB & 100 MB \\ \hline
		des-ede3
			& 0.01 s
			& 0.01 s
			& 0.11 s
			& 9.91 s
			\\ \hline
		
		aes-128-cbc
			& 0.01 s
			& 0.01 s
			& 0.11 s
			& 10.21 s
			\\ \hline

		aes-192-cbc
			& 0.01 s
			& 0.01 s
			& 0.11 s
			& 10.48 s
			\\ \hline
		
		aes-256-cbc
			& 0.01 s
			& 0.01 s
			& 0.11 s
			& 10.37 s
			\\ \hline

		aes-128-ctr
			& 0.01 s
			& 0.01 s
			& 0.14 s
			& 10.39 s
			\\ \hline

		chacha20
			& 0.01 s
			& 0.01 s
			& 0.12 s
			& 9.18 s
			\\ \hline

	\end{tabular}
	\caption{Performance of some symmetric encryption algorithms.}
	\label{tab:symmPerformance}
\end{table}

\subsection*{Digest Algorithms Performances}
\begin{table}[H]
	\centering
	\begin{tabular}{|p{3cm}|p{2cm}|p{2cm}|p{2cm}|p{2cm}|}\hline
	\rowcolor{blue!10}
	& 100 B & 10 kB & 1 MB & 100 MB \\ \hline
	sha256
		&  0.01 s
		&  0.01 s
		&  0.01 s
		&  0.12 s
		\\ \hline
	
	sha512
		&  0.01 s
		&  0.01 s
		&  0.02 s
		&  0.15 s
		\\ \hline

\end{tabular}
\caption{Costs associated with some digest algorithms}
\label{tab:digestAlgorithms}
\end{table}

