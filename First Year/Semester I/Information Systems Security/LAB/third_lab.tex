\section{Third LAB}
\cite{LAB3}

The goal of this laboratory is to experiment with the application of the main cryptographic primitives presented in the previous laboratory. The laboratory uses the OpenSSL (http://www.openssl.org/) open-source library and tools, available for various platforms, including Linux and Windows.

\begin{table}[H]
	\centering
    \begin{tabular}{|p{5cm}|p{4cm}|p{7cm}|}\hline
        \rowcolor{gray!30}
		\textbf{Command} & \textbf{Description} & \textbf{Options} \\ \hline
		\textcolor{Blue}{\texttt{man openssl <command>}} 
			& 
			& 
        \\ \hline

        openssl crl
			& Allows to manipulate a Certificate Revocation List (CRL).
			& -in <file>; -CAfile <file> 
                \newline -text; -noout
        \\ \hline

        openssl x509
            & Sign and view an X.509 certificate.
            & -inform DER|PEM; -outform DER|PEM
                \newline -in <file>; -out <file>
                \newline -noout; -req; -text
        \\ \hline

        openssl ocsp
            & Verify the validity of a certificate.
            & -issuer <file>; -cert <file>
                \newline -url <URL>
                \newline -resp\_text
        \\ \hline

        openssl dhparam
            & Manipulate and generate DH parameters.
            & -in <file>; -out <file>
                \newline -2 (generator); -text
                \newline <num\_bits>
        \\ \hline

        openssl genpkey
            & Generate a private key or key pair
            & -paramfile <file>;
                \newline -out <file>.pem
        \\ \hline

        openssl pkey 
            & Processes public or private keys. They can be converted between various forms and their components printed.
            & -in <key\_file>.pem; -out <file>.pem
                \newline -pubout (pubkey exits); 
                    \newline -pubin (read pubkey)
                \newline -text (output)
        \\ \hline 


    \end{tabular}

    \caption{\texttt{openssl} commands}

\end{table}

\begin{lstlisting}[style=bashStyle]
    # compute a keyed-digest of a file, using SHA-256
    openssl dgst -<hash_algorithm> -binary -hmac <key> -out <msg>.hmac <msg>
    # verify the massage msg, that has been protected with the HMAC calculated above.
    openssl dgst -<hash_algorithm> -binary -hmac <key> -out <file>.verify <msg>
    cmp <file>.verify <file>
\end{lstlisting}


\begin{lstlisting}[style=bashStyle]
    #DIGITAL SIGNATURE
    #sign (hard)
    openssl dgst -<hash_algorithm> -binary <msg> > <msg>.hash
    openssl pkeyutl -sign -in <msg>.hash -inkey <key> -out <msg>.sig
    #verify
    openssl dgst -<hash_algorithm> -binary <msg> > <msg>.hash_verifier
    openssl pkeyutl -verify -in <msg>.hash_verifier -sigfile <msg>.sig -inkey <pub_key> -pubin

    #...but... the easiest way of create/verify signatures with OpenSSL is directly using the command dgst
    #sign (easy)
    openssl dgst -<hash_algorithm> -sign <key> -out <msg>.sig <msg>
    #verify
    openssl dgst -<hash_algorithm> -verify <pub_key> -signature <msg>.sig <msg>

\end{lstlisting}

\begin{lstlisting}[style=bashStyle]
    #KEY ESTABLISHMENT process
    #Alice can use asymmetric encryption to exchange a symmetric key and then use that symmetric key to encrypt the file.
    #Alice has Bob's public key then performs:
    openssl enc -<encrypting_algorithm> -in <file> -out <file>.enc -kfile <key_file>
    openssl pkeyutl -encrypt -in aeskey -pubin -inkey <bob_pub_key> -out <key_file>.enc
    #Alice sends file.enc + key.enc to Bob, which performs:
    #decrypt shared key using Bob's private key
    openssl pkeyutl -decrypt -inkey rsa.key -in aeskey.enc -out aeskey
    #decrypt file using symmetric algorithm and shared key
    openssl enc -d -aes-128-cbc -in chap12.pdf.enc -kfile aeskey -out chap12.pdf
    
\end{lstlisting}

\textbf{Insights:}
\begin{itemize}
    \item To compute a keyed-digest on a message (stored in a file) we use HMAC. On the other hand, Bob in order to verify the keyed-digest must know: the original file, the shared secret key, the digest algorithm (e.g. sha256), the keyed-digest value.
\end{itemize}

\begin{table}[H]
	\centering
    \begin{tabular}{|p{3cm}|p{6cm}|p{7cm}|}\hline
        \rowcolor{gray!30}
		\textbf{Command} & \textbf{Description} & \textbf{Options} \\ \hline
        cmp 
			& Compare two files byte by byte
			& -l (verbose); -n <bytes\_limit>
                \newline -s(ilent)
                \newline -b (diff bytes)
        \\ \hline

        xxd 
            \newline 
            \newline Non-printable characters are replaced with a dot "."
            & Creates  a  hex  dump of a given file or standard input.  It can also convert a hex dump back to its original binary form.
            & -b (bits dump)
        \\ \hline 

        sqlite3 
            & Terminal-based front-end to the SQLite library.
            & -header; -column
        \\ \hline


    \end{tabular}

    \caption{Utility commands}

\end{table}

\section{Digital Certificates}
\subsection{X.509v3 Certificate}

\begin{lstlisting}[style=bashStyle]
    #view the content of an X.509 certificate
    openssl x509 -in <cert_file> -text -noout

    #the Certificate Revocation List (CRL) can be downloaded from the URL provided in the certificate (check the CRL Distribution Points (CDP) field in the certificate details)
    #view CRL content:
    openssl crl -inform DER|PEM -in <crl_file> -text

    #same for the OCSP responder (under Authority Information Access field)

    #check the status of the certificate
    openssl ocsp -issuer <file>.pem -cert <file>.pem -url <URL> -resp_text
\end{lstlisting}

\section{Key Exchange with DH Algorithm}

\begin{center}
    (Diffie-Hellman Algorithm)
\end{center}

The Diffie-Hellman (DH) algorithm is one of the foundational methods for secure key exchange. It enables two parties to establish a shared secret over an insecure communication channel without transmitting the secret itself.

\begin{quotation}
    Now, let’s assume that Alice and Bob want to use the Diffie-Hellman algorithm to agree on a common secret
key. You will use OpenSSL commands for this purpose.
\end{quotation}

\begin{lstlisting}[style=bashStyle]
    #generate public parameters (used by both peers) for DH
    openssl dhparam -out dhparams.pem -2 1024
    #in order to see them
    openssl dhparam -in dhparams.pem -text

    #Alice and Bob use the parameters to generate their own DH pair (public and private).
    openssl genpkey -paramfile dhparams.pem -out <user>dhkey.pem
    #in order to read them
    openssl pkey -in <user>dhkey.pem -text

    #Alice and Bob have to exchange their own public keys
    #thus, each of them will have to extract DH pub keys and store them
    openssl pkey -in <user>dhkey.pem -pubout -out <user>dhPUBkey.pem
    #in order to view them
    openssl pkey -pubin -in <user>dhPUBkey.pem -text

    #Now, Alice and Bob can generate their shared key
    openssl pkeyutl -derive -inkey <curr-user>dhkey.pem -peerkey <user2>dhPUBkey.pem -out <curr-user>_secret.bin 
    
    #comparing the two generated keys you should not receive any output
    cmp -b <curr-user>_secret.bin <user2>_secret.bin

    #in order to see that the two keys are identical
    xxd <user>_secret.bin

\end{lstlisting}

\section{Asymmetric Challenge Response Authentication}
In this exercise, you’ll use the RSA functions (and the pkeyutl command) to implement a challenge-response protocol.

Form two groups (Alice and Bob) and proceed in the following way:
\begin{itemize}
    \item Alice sends Bob her own public key rsa.pubkey.alice (if Alice and Bob are on the same PC it is not necessary to transfer any data, Alice just tells Bob the correct file on the file system to be used).
    \item Bob generates a random string in the file random and encrypts it with Alice’s public key: the result obtained is the file challenge. Run the following OpenSSL commands required to perform this step, by using rand and pkeyutl
    \begin{lstlisting}[style=bashStyle]
openssl rand -out random 20
openssl pkeyutl -encrypt -pubin -inkey rsa.pubkey.alice -in random -out challenge
    \end{lstlisting}
    \item Bob sens the file "challenge" to Alice.
    \item Alice decrypts the challenge in the file response and sends it (or makes it available) to Bob. Use the following pkeyutl command to complete this step (the response to the challenge is saved in the file response):
    \begin{lstlisting}[style=bashStyle]
openssl pkeyutl -decrypt -inkey rsa.key.alice -in challenge -out response
    \end{lstlisting}
\end{itemize}

\section{Authentication with Passwords}
Storing passwords in clear within a database is a bad practice that leads to serious security concerns. This
practice is equivalent to writing them down on a piece of digital paper. If an attacker breaks into the database,
or rogue system administrators improperly access it, all credentials would be directly available. This will grant
him access to all the users’ accounts. This section aims at showing, at first, how to mitigate this issue, and then
how to perform a dictionary attack over the password hashes.

\subsection{Password Hashing}
\begin{lstlisting}[style=bashStyle]
#fast way to compute a digest for a string
encho -n "mypassword" | openssl dgst -sha256
#the option -n is necessary to exclude \n (do not output the trailing newline)
#or use
openssl dgst -<hash_algorithm> -out <hashed_text> <ptext>

\end{lstlisting}

\subsection{Dictionary Attack}

In this section, we will demonstrate how to perform a dictionary attack. A dictionary attack involves using a predefined list of potential passwords—known as a dictionary—to quickly compare a given password hash (or multiple hashes) against the entries in this list. The hash could be obtained from a security breach or stolen from a database.

If the cracking software finds a match between the target hash and an entry in the dictionary, the attacker can successfully recover the original password. This technique is effective when the password is relatively common or simple, as the dictionary will likely contain the most frequently used passwords.

% ------




Use this Python code:
\begin{lstlisting}[style=pythonStyle]
import time
import sqlite3

test_hash = input("Please insert the password hash you'd like to test: ")
dict_db = input("Filename containing the db file: ")
start_time = time.time()

# DB connection
con = sqlite3.connect(dict_db)
c = con.cursor()

# Hash lookup
c.execute('''
          SELECT pwd FROM dict_attack WHERE hash = ?
          ''', [test_hash])

# Retrieve the password if the dictionary contains the hash
result = c.fetchone()

if result != None:
    print("The password has been cracked --> "+result[0])
else:
    print("The attack failed!")

print("Time elapsed: %s s" % (time.time() - start_time))

\end{lstlisting}

\begin{lstlisting}[style=bashStyle]
python3 hash_dict.py 
#will ask for an hash to resolve, try using
9fd13a8c6c17da6b9ed242f788efd8fe9fd5143e3444091b8cc5aa6a9c263114
#will ask to locate the database (we are using sqlite3), set the name
#The output in our case is:
The password has been cracked --> fishes
Time elapsed: 0.002219676971435547

#in order to inspect the dicitonary content: 
sudo apt update && sudo apt install sqlite3 -y
sqlite3 dictionary.db -header -column "SELECT * FROM dict attack LIMIT 10;"
\end{lstlisting}

\subsection{Password Salting}
In this exercise, we will explore the salting mechanism used commonly to protect from dictionary attacks. Form a group of two hosts, Alice and Bob.

On Bob (acting as a server) open a terminal and run the following commands:
\begin{lstlisting}[style=bashStyle]
sudo adduser test1 --disabled-password 
#(press enter for default options)
sudo mkpasswd --method=sha512crypt --salt=coolsalt 1234
#you should see the following output
$6$coolsalt$XMeYB41McYDfApgGicyIRK7JC4I.wThWxLwwOSbW7HMFXJZJFxMdsShTIsxoiy/yG2BKqqDIRH2Aasf/XDWks/
#open the /etc/shadow file
#In the entry corresponding to the user test1, modify the row in the following way (copy and paste the row):
test1:$6$coolsalt$XMeYB41McYDfApgGicyIRK7JC4I.wThWxLwwOSbW7HMFXJZJFxMdsShTIsxoiy/yG2BKqqDIRH2Aasf/XDWks/:19674:0:99999:7:::
\end{lstlisting}

\subsection*{Shadow File}
\begin{center}
    (/etc/shadow)
\end{center}
Shadow file on a Linux system is used to store password hashes and other security-related information about system users. It is an essential file in the system’s authentication mechanism. This file is readable only by the root user for security reasons.

    Each line in the /etc/shadow file corresponds to a user account and contains fields separated by colons (:). Here’s the general format:
    \begin{lstlisting}[style=bashStyle]
username:password_hash:last_changed:min_days:max_days:warn_days:inactive_days:expire_date:reserved
    \end{lstlisting}


Explanation of \texttt{/etc/shadow} Fields:

\begin{enumerate}
    \item \textbf{password\_hash:}
    \begin{itemize}
        \item Contains the hashed password.
        \item Asterisk (\texttt{*}) or exclamation mark (\texttt{!}) indicates a disabled account.
        \item Empty means no password is required.
        \item Prefix: \(\$6\$\) \textrightarrow sha-512, \(\$5\$\) \textrightarrow sha-256, \(\$1\$\) \textrightarrow MD5, \(\$y\$\) \textrightarrow Yescrypt.
    \end{itemize}

    \item \textbf{last\_changed:}
    \begin{itemize}
        \item The number of days since January 1, 1970, when the password was last changed.
    \end{itemize}

    \item \textbf{min\_days:}
    \begin{itemize}
        \item Minimum number of days required between password changes (\texttt{0} = no restrictions).
    \end{itemize}

    \item \textbf{warn\_days:}
    \begin{itemize}
        \item Number of days before password expiration when the user is warned.
    \end{itemize}

    \item \textbf{inactive\_days:}
    \begin{itemize}
        \item Number of days after password expiration before the account is disabled.
    \end{itemize}

    \item \textbf{expire\_date:}
    \begin{itemize}
        \item Account expiration date (in days since January 1, 1970). Empty means no expiration.
    \end{itemize}
\end{enumerate}