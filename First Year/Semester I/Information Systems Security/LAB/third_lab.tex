\section{Third LAB}
\cite{LAB3}

The goal of this laboratory is to experiment with the application of the main cryptographic primitives presented in the previous laboratory. The laboratory uses the OpenSSL (http://www.openssl.org/) open-source library and tools, available for various platforms, including Linux and Windows.

\begin{table}[H]
	\centering
    \begin{tabular}{|p{5cm}|p{4cm}|p{7cm}|}\hline
        \rowcolor{gray!30}
		\textbf{Command} & \textbf{Description} & \textbf{Options} \\ \hline
		\textcolor{Blue}{\texttt{man openssl <command>}} 
			& 
			& 
        \\ \hline

        openssl crl
			& Allows to manipulate a Certificate Revocation List (CRL).
			& -in <file>; -CAfile <file> 
                \newline -text; -noout
        \\ \hline

        openssl x509
            & Sign and view an X.509 certificate.
            & -inform DER|PEM; -outform DER|PEM
                \newline -in <file>; -out <file>
                \newline -noout; -req; -text
        \\ \hline

        openssl ocsp
            & Verify the validity of a certificate.
            & -issuer <file>; -cert <file>
                \newline -url <URL>
                \newline -resp\_text
        \\ \hline

        openssl dhparam
            & Manipulate and generate DH parameters.
            & -in <file>; -out <file>
                \newline -2 (generator); -text
                \newline <num\_bits>
        \\ \hline

        openssl genpkey
            & Generate a private key or key pair
            & -paramfile <file>;
                \newline -out <file>.pem
        \\ \hline

        openssl pkey 
            & Processes public or private keys. They can be converted between various forms and their components printed.
            & -in <key\_file>.pem; -out <file>.pem
                \newline -pubout (pubkey exits); 
                    \newline -pubin (read pubkey)
                \newline -text (output)
        \\ \hline 


    \end{tabular}

    \caption{\texttt{openssl} commands}

\end{table}

\begin{lstlisting}[style=bashStyle]
    # compute a keyed-digest of a file, using SHA-256
    openssl dgst -<hash_algorithm> -binary -hmac <key> -out <msg>.hmac <msg>
    # verify the massage msg, that has been protected with the HMAC calculated above.
    openssl dgst -<hash_algorithm> -binary -hmac <key> -out <file>.verify <msg>
    cmp <file>.verify <file>
\end{lstlisting}


\begin{lstlisting}[style=bashStyle]
    #DIGITAL SIGNATURE
    #sign (hard)
    openssl dgst -<hash_algorithm> -binary <msg> > <msg>.hash
    openssl pkeyutl -sign -in <msg>.hash -inkey <key> -out <msg>.sig
    #verify
    openssl dgst -<hash_algorithm> -binary <msg> > <msg>.hash_verifier
    openssl pkeyutl -verify -in <msg>.hash_verifier -sigfile <msg>.sig -inkey <pub_key> -pubin

    #...but... the easiest way of create/verify signatures with OpenSSL is directly using the command dgst
    #sign (easy)
    openssl dgst -<hash_algorithm> -sign <key> -out <msg>.sig <msg>
    #verify
    openssl dgst -<hash_algorithm> -verify <pub_key> -signature <msg>.sig <msg>

\end{lstlisting}

\begin{lstlisting}[style=bashStyle]
    #KEY ESTABLISHMENT process
    #Alice can use asymmetric encryption to exchange a symmetric key and then use that symmetric key to encrypt the file.
    #Alice has Bob's public key then performs:
    openssl enc -<encrypting_algorithm> -in <file> -out <file>.enc -kfile <key_file>
    openssl pkeyutl -encrypt -in aeskey -pubin -inkey <bob_pub_key> -out <key_file>.enc
    #Alice sends file.enc + key.enc to Bob, which performs:
    #decrypt shared key using Bob's private key
    openssl pkeyutl -decrypt -inkey rsa.key -in aeskey.enc -out aeskey
    #decrypt file using symmetric algorithm and shared key
    openssl enc -d -aes-128-cbc -in chap12.pdf.enc -kfile aeskey -out chap12.pdf
    
\end{lstlisting}

\textbf{Insights:}
\begin{itemize}
    \item To compute a keyed-digest on a message (stored in a file) we use HMAC. On the other hand, Bob in order to verify the keyed-digest must know: the original file, the shared secret key, the digest algorithm (e.g. sha256), the keyed-digest value.
\end{itemize}

\begin{table}[H]
	\centering
    \begin{tabular}{|p{3cm}|p{6cm}|p{7cm}|}\hline
        \rowcolor{gray!30}
		\textbf{Command} & \textbf{Description} & \textbf{Options} \\ \hline
        cmp 
			& Compare two files byte by byte
			& -l (verbose); -n <bytes\_limit>
                \newline -s(ilent)
                \newline -b (diff bytes)
        \\ \hline

        xxd 
            \newline 
            \newline Non-printable 