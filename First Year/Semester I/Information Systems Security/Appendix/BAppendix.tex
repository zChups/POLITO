\chapter{Appendix B: Attack Vectors}

\begin{table}[H]
    \centering
    \begin{tabular}{|p{3cm}|p{12cm}|}\hline
    \rowcolor{blue!10}
    \textbf{Vector} 
        & \textbf{Description}  
    \\ \hline

    Trojan
        & Appears as legitimate software or a harmless file to deceive users into installing it. Contains a dangerous payload and is often used to create a Man At The End or a Man In The Browser.
    \\ \hline
    
    Virus
        & Attaches itself to legitimate files or programs and spreads by infecting other files on a system. Is propagated by humans (often involuntarily).
    \\ \hline

    Worm
        & Self-replicating malware that spreads across networks without requiring user intervention. Unlike a virus, a worm does not need to attach itself to a file or program.
    \\ \hline

    Backdoor
        & Unauthorized access point, it allows an attacker to remotely control the system or retrieve sensitive information without the knowledge of the legitimate users or administrators.
    \\ \hline

    Rootkit
        & Designed to gain unauthorized access to a system and maintain privileged control (root access) while hiding its presence from users and security software.
    \\ \hline

    Potentially Unwanted Applications (PUAs)
        & Software programs that are not inherently malicious but may negatively impact a user’s system or experience. They often include programs like adware, toolbars, or system optimizers that may be bundled with other software. While PUAs typically don’t cause direct harm like viruses or malware, they can degrade system performance, display excessive ads, or compromise privacy by collecting personal data without clear consent. 
    \\ \hline

    Ransomware 
        & Encrypts a user’s files or locks them out of their system, rendering the data inaccessible. The attacker then demands a ransom.
    \\ \hline



\end{tabular}
\caption{Attack Vectors}
\label{tab:attackVectors}
\end{table}